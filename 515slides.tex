\documentclass[12pt]{beamer}
\usetheme{CambridgeUS}
\usecolortheme{dolphin}
\usepackage[utf8]{inputenc}
\usepackage{amsmath}
\usepackage{amsfonts}
\usepackage{amssymb}
\usepackage{graphicx}
\usepackage{mathtools}
\usepackage{multimedia}
\usepackage[mathscr]{euscript}
\usepackage{xcolor}



% Useful macro for circling numbers
\usepackage{tikz}
\newcommand*\circled[1]{\tikz[baseline=(char.base)]{
  \node[shape=circle,draw,inner sep=2pt] (char) {#1};}}

\DeclarePairedDelimiter{\abs}{\lvert}{\rvert}
\DeclarePairedDelimiter{\norm}{\lVert}{\rVert}

\newcommand{\R}{\mathbb{R}}
\newcommand{\C}{\mathbb{C}}
\newcommand{\Tau}{\mathcal{T}}
\newcommand{\AMz}{\underset{z}{arg\, min}}
\newcommand{\prox}{\text{prox}}

\newcommand{\worm}{WWWWWoooooooRRRRRmmmmmm}

\author[Rudy,Maass,Molloy,Mueller]{Samuel Rudy, Kelsey Maass, Riley Molloy, and Kevin Mueller}
\title[Denoising and Deblurring]{Image Denoising and Deblurring \\ \normalsize Applied Math 515 Final Project}
\setbeamercovered{transparent} 
%\institute{University of Washington} 
\date{March 12 2016} 
\subject{Amath 575} 

\iffalse
To d:
-Discuss different fidelity functions
-Discuss different regularizers
-Talk about optimization for wavelet method
-Talk about Frobenius norm denoising
-TV reg for denoising and deblurring
-Give lots of examples
-Talk about caveats of each method (parameters are a pain in the ass)
-bibliography
\fi



\begin{document}

\begin{frame}
\titlepage
\end{frame}


\begin{frame}{Contents}
\tableofcontents
\end{frame}

\section{Motivation}

\begin{frame}{Image Denoising and Deblurring}
\begin{center}
\includegraphics[scale=0.55]{bad_image.png}
\hspace{1 cm}
\includegraphics[scale=0.55]{good_image.png} 
\end{center}
\end{frame}

\begin{frame}{Mathematical Formulation}

\begin{exampleblock}{Artificial Blur/Noise}
\begin{itemize}
\item Blur added via convolution with Gaussian kernel.
\item Gaussian or Student-t noise added to blurred image.
\end{itemize}
\end{exampleblock}

\begin{exampleblock}{Image Recovery}
Find an image $x$ balancing two properties
\begin{enumerate}
\item Convolution of $x$ with blur kernel is similar to $b$
\item Some measure of noise on $x$ is regularized
\end{enumerate}
\end{exampleblock}

This slide kind of sucks as is :)  Anyone wanna fix it?

\end{frame}

\section{De(noise/blur)ing Objective Functions}

\begin{frame}{A General Loss Function}

$$
L_b(x) = \underbrace{f(Ax-b)}_{\text{Fidelity Term}} + \underbrace{\lambda g(x)}_{\text{Noise Regularization}} + \underbrace{\delta(x | [0,1])}_{\text{Range of Pixel Values}}
$$

\begin{exampleblock}{Fidelity Term}
\vspace{-5 mm}
\begin{align*}
f &= \left\{ \begin{aligned}
&\|\cdot \|_F^2 \\
& h_\gamma (\cdot)\\
& \gamma^{-1} \log ( \cosh (\gamma \cdot ))
\end{aligned}\right.
\end{align*}
\end{exampleblock}

\begin{exampleblock}{Regularization Term}
\vspace{-5 mm}
\begin{align*}
g &= \left\{ \begin{aligned}
&TV(x) \\
& \| Wx\|_1
\end{aligned}\right.
\end{align*}
\end{exampleblock}

\end{frame}

\begin{frame}{Regularizer}

Talk about choice of g
Haar, FFT
What is TV?

Show two different definitions of TV from paper.

\end{frame}

\begin{frame}{Fidelity Function}

\begin{minipage}{0.45\textwidth}

\begin{center}
\vspace{-2 mm}
\includegraphics[scale=0.35]{gaussian_noise.png} \\
\includegraphics[scale=0.35]{student_t_noise.png} 
\end{center}

\end{minipage} \hfill
\begin{minipage}{0.52\textwidth}

What is Ax - b
Why use different functions than frobenius norm?  Use pictures as motivation

\end{minipage}

\end{frame}

\section{Optimization with Wavelet Regularization}

\begin{frame}{Kelsey's stuff here}

\end{frame}

\section{Optimization with Total Variation Regularization}

\begin{frame}{Total Variation Regularization}

\begin{exampleblock}{Loss Function}
$$
L_b(x) = f(Ax-b) + \lambda \|x\|_{TV} + \delta(x | [0,1])
$$
\end{exampleblock}

\begin{exampleblock}{Proximal Gradient Step}
\begin{align*}
x^{k+1} &= \prox_{\mathcal{L}^{-1}(\lambda \|\cdot \|_{TV} + \delta_{[0,1]})} (\underbrace{x^k - \mathcal{L}^{-1} A^T\nabla f (Ax^k - b)}_{u^k}) \\
&= \AMz \left( \|u^k - z\|_F^2 + \lambda \|z\|_{TV} + \delta(z | [0,1]) \right) \\
&= P_{[0,1]}  \left( \AMz \left( \|u^k - z\|_F^2 + \lambda \|z\|_{TV} \right) \right)
\end{align*}
\end{exampleblock}

\end{frame}

\begin{frame}{Dual Form of Total Variation}

\begin{exampleblock}{A Few Definitions}
weee
\end{exampleblock}

\begin{exampleblock}{Total Variation}
blarg
\end{exampleblock}

\end{frame}

\begin{frame}{Dual Form of TV Denoising with $\|\cdot \|_F^2$}

\end{frame}

\begin{frame}{Optimization of Dual Form}

\end{frame}

\section{Discussion}

\section*{References}
\begin{frame}{Questions?}
\begin{thebibliography}{10}    
\setbeamertemplate{bibliography item}[online]
\bibitem{code} Codes used to generate figures
\url{https://github.com/snagcliffs/Amath575project}

\beamertemplatebookbibitems % This gives it a nice book symbol
\bibitem{Guck} Guckenheimer, J., Holmes, P. \textit{Nonlinear Oscillations, Dynamical Systems, and Bifurcations of Vector Fields}. Springer-Verlag, 1983. Print.

\beamertemplatearticlebibitems % and an article symbol
\bibitem{brazil}  Oliveira, D., Leonel, E. (2008) \textit{Braz. J. Phys.} 38(1):62-64

\bibitem{corrdim} Grassberger, P., Procaccia, I. (1983) \textit{Phys. Rev. Letters.} 50(5):346-349

\end{thebibliography}
\end{frame}

\end{document}
