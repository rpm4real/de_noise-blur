\documentclass[10pt,a4paper]{article}
\usepackage{nips15submit_e}
\usepackage{amsmath}
\usepackage{mathtools}
\usepackage{amsfonts}
\usepackage{amssymb}
\usepackage{graphicx}
\usepackage{float}
\usepackage{hyperref}
%\usepackage[
%backend=bibtex,
%sorting=unsrt
%]{biblatex}
%\addbibresource{projectsources.bib}
\usepackage{xcolor}

\usepackage[paperwidth=8.5in,paperheight=11in,centering,hmargin=1in,vmargin=1in]{geometry}
\nipsfinalcopy

%\DeclarePairedDelimiter{\norm}{\lVert}{\rVert}
\usepackage{physics}
\newcommand{\prox}{\mathrm{prox}}
\newcommand{\R}{\mathbb{R}}
\newcommand{\C}{\mathbb{C}}
\newcommand{\Tau}{\mathcal{T}}

\begin{document}
\title{Image Deblurring and Denoising for Various Fidelity Terms and Regularizers}
\author{
Kelsey Maass, ~Samuel Rudy, ~Kevin Mueller, ~and Riley Molloy\\
University of Washington\\
}

\maketitle

Abstract?

% Introduction
\section{Introduction}

Image denoising and deblurring problems can be modeled with the linear system
\begin{equation}
Ax + w = b,
\end{equation}
where $b$ is the observed image, $w$ is some unknown noise, and $x$ is the true image we would like to recover. For a blurred image we let $A$ represent the blurring process, and for noisy images we let $A$ be the identity matrix. For many inverse problem applications the least squares approximation, 
\begin{equation}
\hat{x}_{LS} = \arg\min_x \| Ax - b \|_2^2 ,
\end{equation}
gives a reasonable solution. Unfortunately, this approach doesn't provide any new information for the denoising problem, and for image deblurring problems $A$ is often ill-conditioned, so the solution is contaminated by round-off error and amplified noise. To illustrate this, we consider the pepper image below. If we blur the image and compute the naive solution $x = A^{-1}b$, the result is obscured by round-off error in the form of ringing artifacts. The result is even worse when we add noise.

\begin{figure}[H]
\centering
\includegraphics[width=2in]{../figures/fig1}
\includegraphics[width=2in]{../figures/fig2}
\includegraphics[width=2in]{../figures/fig3} \\[1ex]
\includegraphics[width=2in]{../figures/fig1}
\includegraphics[width=2in]{../figures/fig4}
\includegraphics[width=2in]{../figures/fig5}
\caption{\small In the top row we observe the naive solution applied to a blurred image, including the true image $x$ (left), the blurred image $b$ (middle), and the recovered image $A^{-1}b$. On the bottom row, we see the effects of adding noise illustrated by the true image $x$ (left), the blurred and noisy image $b-w$ (middle), and the recovered image $A^{-1}(b-w)$.}
\end{figure}

In order to stabilize the solution, a variety of regularizers can be used which exploit features of the true image such as smoothness or sparsity in a wavelet domain. This results in the problem formulation
\begin{equation}
\hat{x} = \arg\min_{x} \| Ax - b \|_2^2 + \lambda R(x),
\end{equation}
where the parameter $\lambda > 0$ is chosen to balance the tradeoff between fidelity to the model and the assumed feature. In this paper we compare two choices for the regularization term, 
\begin{align}
&\text{Total variation regularization:} &\hat{x}_{TV} = \arg\min_x \| Ax - b \|_2^2 + 2\lambda \| x \|_{TV} , \\
&\text{and } \ell_1 \text{ regularization:} &\hat{x}_{\ell_1} = \arg\min_x \| Ax - b \|_2^2 + \lambda \| Wx \|_1  ,
\end{align}

which are explored by Beck and Teboulle in \cite{TV} and \cite{FISTA} respectively.

It is also known that the quadratic penalty is extremely sensitive to outliers. While this does not pose problems for images with Gaussian noise, which may result from atmospheric turbulence, it becomes an issues for images with noise from heavier tailed distributions, such as the Student's t-distribution, which could result from a dirty camera lens. Therefore we also consider different choices for the fidelity term which are more robust to outliers than the quadratic penalty, including the Huber norm
\begin{equation} \label{huber}
h_{\gamma}(x) = \min_y \frac{1}{2} \| x - y \|_2^2 + \gamma \| y \|_1,
\end{equation}
and the function
\begin{equation} \label{log_cosh}
g_{\gamma}(x) = \frac{1}{\gamma} \sum_i \log\left(\cosh\left(\gamma x_i\right)\right).
\end{equation}

% Problem Formulation
\section{Problem Formulation}
The general approach of to an image deblurring or denoising problem is the minimization of a loss function of the form \cite{DeblurBook}
\begin{equation} \label{general}
 \min_x f(\mathcal{A}(x) -b) + g(x),
\end{equation}
where $\mathcal{A}(x)$ is a convolution of the desired image $x$ with a gaussian kernel creating a blur,  $f : \R^{m \times n} \rightarrow [0, \infty)$ represents some continuous measure of distance between the corrupt image $b$ and the desired image $x$, and $g:\R^{m \times n} \rightarrow [0,\infty)$ is some regularization on the allowed amount of noise in $x$. The term $f(\mathcal{A}(x)-b)$ is referred to as the \emph{fidelity term}, and the term $g(x)$ is referred to as the \emph{regularization}. We can represent the blurring convolution $\mathcal{A}(\cdot): \R^{m \times n} \rightarrow \R^{m \times n}$ by left matrix multplication with some matrix $A$, so we do so for convenience from here forward. In addition, since we require some pixel value $x_{ij} \in [0,1]$, we can add an additional term to the loss function as
\begin{equation} \label{loss}
 L_b(x) = f(Ax-b) + g(x) + \delta(x | [0,1] ),
\end{equation}
where $\delta(x | [0,1])$ is an indiciator function for the unit interval. 

A simple way to solve for the real image $x$ give $g(x) = 0$ is to apply the gradient descent algorithm,

\begin{equation}
x_k = x_{k-1} - t_k \nabla f(x_{k-1}),
\end{equation}

for suitable stepsize $t_k > 0$. When $g$ is nontrivial and nonsmooth (the usual case), we can rewrite this as a proximal mapping,

\begin{equation}
\prox_g(x_{k-1} - \nabla f(x_{k-1})) = \min_x \left( \frac{1}{2}||x-x_{k-1} - \nabla f(x_{k-1})||_2^2  + \lambda g(x)\right)
\end{equation}

The advantage of this form is that many nonsmooth regularizers have a close formed solution for their $\prox$.

% 1-Norm Wavelet Regularization
\subsection{Wavelet Regularization}

The wavelet regularization deblurring model, as seen in \cite{FISTA}, can be formulated in the form of \eqref{general} as 
\begin{equation} \label{wavelet_orig}
\min_{x} \norm{ Ax - b }_F^2 + \lambda \norm{Wx}_1 ,
\end{equation}
where $\norm{\cdot}_F$ is the Frobenius norm, $\lambda>0$ is a regularization parameter, and $W$ corresponds to a wavelet trasform of a given wavelet type. The primary motivation for this approach is that most images have a sparse representation in the wavelet domain that can be easily exploited by $\ell_1$ regularization. 

Our approach considers a more general problem
\begin{equation} \label{wavelet}
\min_{x} f(Ax - b ) + \lambda \norm{Wx}_1,
\end{equation}
where $f: \R^{m \times n} \rightarrow [0,\infty)$ is some Lipschitz differentiable functional which gives a measurement of the size of the fidelity term. We specifically work with the cases 
\begin{equation} \label{fidelities}
f(x) = \begin{cases}
\norm{x}_F^2 \\
h_\gamma(x) \\
g_\gamma(x),
\end{cases}
\end{equation}
where $h_\gamma$ and $g_\gamma$ are as in \eqref{huber} and \eqref{log_cosh} respectively. 

% TV Regularization
\subsection{Total Variation Regularization}
The usual Total-Variation deblurring model, as seen in \cite{TV} can be formulated in the form of \eqref{general} as 
\begin{equation} \label{tv_orig}
\min_{x} \norm{ Ax - b }_F^2 + 2 \lambda \mathrm{TV}(x),
\end{equation}
where $\norm{\cdot}_F$ is the Frobenius norm, $\lambda>0$ is a regularization parameter, and $\mathrm{TV}(x)$ is the Total-Variation semi-norm. Two choices similar choices exist for the TV-norm: the so-called isotropic type, and the $\ell_1$ type. In this work, we work exclusively with the $\ell_1$-based TV-norm, defined as 
$$ TV_{\ell_1}(x) = \sum_{i=1}^{m-1} \sum_{j=1}^{n-1} \left( \abs{x_{i,j}  - x_{i+1,j} } + \abs{ x_{i,j} - x_{i,j+1}  } \right) + \sum_{i=1}^{m-1} \abs{ x_{i,n} - x_{i+1,n} } + \sum_{j=1}^{n-1} \abs{ x_{m,j} - x_{m,j+1 } },$$ for $x \in \R^{m \times n},$ and where the reflexive boundary conditions
\begin{align*}
x_{m+1,j} - x_{m,j} &= 0, \textrm{ for all }j \\
 x_{i,n+1} - x_{i,n} &= 0, \textrm{ for all }i
\end{align*}
are assumed. Our approach considers a more general problem
\begin{equation} \label{tv}
\min_{x} f(Ax - b ) + 2 \lambda \mathrm{TV}_{\ell_1}(x),
\end{equation}
where $f: \R^{m \times n} \rightarrow [0,\infty)$ is some Lipschitz differentiable functional which gives a measurement of the size of the fidelity term. Again, we work with three cases of $f$ as seen in \eqref{fidelities}.


% Algorithms
\section{Algorithms}

\emph{here we can explain prox-gradient approach for each specific case}

\subsection{Wavelet Regularization}

\subsection{Total Variation Regularization}

% Examples
\section{Examples}

\subsection{Wavelet Regularization}


\begin{figure}[H]
\begin{center}
\rotatebox{90}{\hspace{1cm} Original Image \hspace{3cm} Frobenius Loss}
\includegraphics[width = 0.75\textwidth]{../figures/waveletGaussH.pdf} 
\rotatebox{270}{\hspace{-9cm} Blurred and Noisy \hspace{2.6cm} Huber Loss}
\end{center}
\caption{$\ell_1$ Wavelet algorithm performed on an image with Gaussian noise (magnitude $1 \times 10^{-3}$) for the two different fidelity terms of Frobenius and Huber loss. This method used the Haar wavelet for deblurring.}
\label{waveletH_gauss}
\end{figure}

\begin{figure}[H]
\begin{center}
\rotatebox{90}{\hspace{1cm} Original Image \hspace{3cm} Frobenius Loss}
\includegraphics[width = 0.75\textwidth]{../figures/waveletGaussD.pdf} 
\rotatebox{270}{\hspace{-9cm} Blurred and Noisy \hspace{2.6cm} Huber Loss}
\end{center}
\caption{$\ell_1$ Wavelet algorithm performed on an image with Gaussian noise (magnitude $1 \times 10^{-3}$) for the two different fidelity terms of Frobenius and Huber loss. This method used the Daubechies wavelet for deblurring.}
\label{waveletD_gauss}
\end{figure}

\begin{figure}[H]
\begin{center}
\rotatebox{90}{\hspace{1cm} Original Image \hspace{3cm} Frobenius Loss}
\includegraphics[width = 0.75\textwidth]{../figures/waveletStudentH.pdf} 
\rotatebox{270}{\hspace{-9cm} Blurred and Noisy \hspace{2.6cm} Huber Loss}
\end{center}
\caption{$\ell_1$ Wavelet algorithm performed on an image with Student's t noise (magnitude $1 \times 10^{-4}$) for the two different fidelity terms of Frobenius and Huber loss. This method used the Haar wavelet for deblurring.}
\label{waveletH_student}
\end{figure}

\begin{figure}[H]
\begin{center}
\rotatebox{90}{\hspace{1cm} Original Image \hspace{3cm} Frobenius Loss}
\includegraphics[width = 0.75\textwidth]{../figures/waveletStudentD.pdf} 
\rotatebox{270}{\hspace{-9cm} Blurred and Noisy \hspace{2.6cm} Huber Loss}
\end{center}
\caption{$\ell_1$ Wavelet algorithm performed on an image with Student's t noise (magnitude $1 \times 10^{-4}$) for the two different fidelity terms of Frobenius and Huber loss. This method used the Daubechies wavelet for deblurring.}
\label{waveletD_student}
\end{figure}

\subsection{Total Variation Regularization}

We demonstrate some results of denoising and deblurring with the Total Variation regularization on real images. As previously shown, each image was given a small amount of either Gaussian or Student's t noise and blurred. Two different fidelity functions were used: the squared Frobenius norm, and the Huber norm. Figure \ref{tv_gauss} demonstrates the results using Gaussian noise, and Figure \ref{tv_student} demonstrates the results using Student's t noise with one degree of freedom. For the Gaussian noise, we set $\lambda = .001$, and $\gamma = .02$ (see TV-regularization in Problem Formulation). For Student's t noise, we set $\lambda = .002$, and $\gamma = .02$.

\begin{figure}[H]
\begin{center}
\rotatebox{90}{\hspace{1cm} Original Image \hspace{3cm} Frobenius Loss}
\includegraphics[width = 0.75\textwidth]{../figures/gaussian_peppers.png} 
\rotatebox{270}{\hspace{-9cm} Blurred and Noisy \hspace{2.6cm} Huber Loss}
\end{center}
\caption{Total Variation algorithm performed on an image with Gaussian noise (magnitude $1 \times 10^{-3}$) for the two different fidelity terms of Frobenius and Huber loss.}
\label{tv_gauss}
\end{figure}

\begin{figure}[H]
\begin{center}
\rotatebox{90}{\hspace{1cm} Original Image \hspace{3cm} Frobenius Loss}
\includegraphics[width = 0.75\textwidth]{../figures/student-t_peppers.png} 
\rotatebox{270}{\hspace{-9cm} Blurred and Noisy \hspace{2.6cm} Huber Loss}
\end{center}
\caption{Total Variation algorithm performed on an image with Student's t noise (magnitude $1 \times 10^{-4}$) for the two different fidelity terms of Frobenius and Huber loss.}
\label{tv_student}
\end{figure}

% Discussion
\section{Discussion and Conclusions}

%\printbibliography[title={Sources}]
\bibliographystyle{unsrt}
\bibliography{sources}


\end{document}